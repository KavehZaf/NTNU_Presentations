%%%%%%%%%%%%%%%%%%%%%
%%%%%%%%%%%  TiTle

%%%%%%%%%%%%%%%%%%%%%%%%%%%%%%
%%%%%%%%%%%%%%%%%%%%%%%%%%%%%%%%%

\usepackage[titletoc,toc,page]{appendix}
\renewcommand{\restoreapp}{}

\usepackage{ragged2e}




\usepackage[T1]{fontenc}
\usepackage[utf8]{inputenc}
\usepackage[english]{babel}






\setlength{\parindent}{4em}
\setlength{\parskip}{1em}
%\renewcommand{\baselinestretch}{2.0}

\usepackage{fourier}
\usepackage{booktabs}
\usepackage{threeparttablex}
\usepackage{longtable} 
\usepackage{lscape} 


\makeatletter
\newcommand\notsotiny{\@setfontsize\notsotiny\@vipt\@viipt}
\makeatother

\makeatletter
\newcommand\footnoteref[1]{\protected@xdef\@thefnmark{\ref{#1}}\@footnotemark}
\makeatother

\usepackage{threeparttable}

\usepackage{hyperref} 
\hypersetup{colorlinks=true,                
    breaklinks=true,                
    urlcolor= blue,                
    linkcolor= blue,                
    bookmarksopen=false,
    filecolor=black,
    citecolor=green,
    linkbordercolor=blue
}
\urlstyle{same}

\usepackage{amsmath} %equations and spacings
\usepackage{eurosym}
\usepackage{lipsum} % acces to random text
%graphics
\usepackage{graphicx}
%\usepackage[demo]{graphicx}
%caption styles
\usepackage[small,bf]{caption}
%\usepackage{caption}
\usepackage[labelformat=simple]{subcaption}
\renewcommand\thesubfigure{(\alph{subfigure})} % see subcaption doc
%<<byman 
\usepackage{colortbl}
\usepackage{xcolor}
\usepackage{color}
\usepackage{sidecap}
%byman>>

%<<byman
%% We want UK-english hyphenation patterns.byman
%\usepackage[british]{babel}
%byman>>

%Remove Paragraph indenting
\setlength{\parindent}{0.0in} 
\setlength{\parskip}{0.1in}

%<<byman
%% Use scalable, PostScript Type 1 versions of the Computer Modern fonts.
\usepackage{type1cm}
%% Replace the standard Computer Modern Typewriter font LaTeX uses
%% for monospace text with the PostScript font Adobe Courier.
\usepackage{courier}
\usepackage[T1]{fontenc}
\usepackage{ae,aecompl}
\usepackage{times}
%% Redefine the font used for the section headings to
%% Helvetica-Narrow Bold.
% for colour shades in tables 
\usepackage{colortbl} 
 %to control the title formating and spaces around the titles
 \usepackage{pdfpages}
\usepackage[compact]{titlesec}
%%byman>>

%spaces under sections
\usepackage[compact]{titlesec}
\titlespacing{\section}{0pt}{*0}{*0}
\titlespacing{\subsection}{0pt}{*0}{*0}
\titlespacing{\subsubsection}{0pt}{*0}{*0}
%fonts
 %\usepackage[T1]{fontenc} %gives light font
 %\usepackage[light,math]{iwona}

%Abbreviations or Acronyms
%\usepackage[intoc]{nomencl}
\usepackage{nomencl}
\renewcommand{\nomname}{List of Abbreviations}
\makenomenclature

\usepackage{csquotes}


%%%%%%%%%%%%%%%%%%%%%%%%%%%%%%%%%%%%%%%%%%%%
%%%%%%%%%%%%%%%%%%%%%%%%%%%%%%%%%%%%%%%%%%%%
%% Import from Sigurd Thesis




%Bibliography
%\usepackage{url}
%\usepackage{natbib}%\usepackage[sectionbib]{natbib}
%\usepackage{chapterbib}
%\bibpunct[:]{(}{)}{;}{a}{}{,} %citation structure
%\bibpunct{(}{)}{,}{a}{}{;} 
%\bibpunct{[}{]}{,}{a}{}{;}
%\bibpunct{(}{)}{;}{a}{}{,} % to follow the A&A style
%\usepackage{chapterbib}
%\usepackage{hyperref}
%\hypersetup{colorlinks=true,citecolor=blue}
%\hypersetup{colorlinks=true,citecolor=blue,linkcolor=blue,urlcolor=blue}

%To change box color around the links and citations, you have these other options :
%\hypersetup{citebordercolor=Violet,filebordercolor=Red,linkbordercolor=Blue}

\newcommand{\cmt}[1]{}%inline comment
  
%tabularx
\usepackage{tabularx}
\usepackage{tabulary}
\usepackage{multirow,longtable} %table on several pages
\usepackage{booktabs}

%enumerate(control spacing)
\newenvironment{packed_enum}{
\begin{enumerate}
	\vspace*{-1}
  	\setlength{\itemsep}{1pt}
  	\setlength{\parskip}{0pt}
  	\setlength{\parsep}{0pt}
}{\end{enumerate}}

%<<byman
\usepackage{multirow}
\usepackage{dcolumn}
\newcolumntype{d}{D{.}{.}{-2}}
\newcommand{\tabincell}[1]{\begin{tabular}{c}#1\end{tabular}}
\newcommand{\tabincellt}[1]{\begin{tabular}{l}#1\end{tabular}}
\newcommand{\merge}[1]{\multicolumn{1}{c}{#1}}

\usepackage{array}
\newcommand{\PreserveBackslash}[1]{\let\temp=\\#1\let\\=\temp}
\newcolumntype{C}[1]{>{\PreserveBackslash\centering}p{#1}}
\newcolumntype{R}[1]{>{\PreserveBackslash\raggedleft}p{#1}}
\newcolumntype{L}[1]{>{\PreserveBackslash\raggedright}p{#1}}

% Used to create tables with rows/cols spanning over se
%% Use fancy chapter headers, with Jos Dingjan's modifications,
%% plus my own tweaks. This style is not part of teTeX,
%% so we are using a local (and renamed) copy.
%\usepackage[Lenny]{fncychapleo}
\usepackage{fncychap}

%% Nicely format and linebreak URLs in the bibliography (and elsewhere).
%%\usepackage{url}
%% Define a new 'leo' style for the package that will use a smaller font.
%\makeatletter
%\def\url@leostyle{%
%  \@ifundefined{selectfont}{\def\UrlFont{\sf}}{\def\UrlFont{\small\ttfamily}}}
%\makeatother
%% Now actually use the newly defined style.
%\urlstyle{leo}
%byman>>

%graphics
\usepackage{epstopdf} %support for eps.
\DeclareGraphicsExtensions{.eps,.ps,.pdf,.png,.jpg}
\usepackage{float} % figure placing [H]

%landscape Option
\usepackage{lscape} %Left down
%usepackage{pdflscape} %Left up)

\usepackage[draft]{todonotes}   % notes showed (JJUNJU)
% Select what to do with command \comment:  
% \newcommand{\comment}[1]{}  %comment not showed
\newcommand{\comment}[1]
{\par {\bfseries \color{red} #1 \par}} %comment showed

%Section Numbering and TOC depth
\setcounter{secnumdepth}{2}
\setcounter{tocdepth}{3}

%chapter biblio
%\usepackage{chapterbib}

%Drawing flow charts
\usepackage{tikz}
\usetikzlibrary{shapes,arrows}
\tikzstyle{decision} = [diamond, draw, fill=blue!20, 
    text width=4.5em, text badly centered, node distance=3cm, inner sep=0pt]
\tikzstyle{process} = [rectangle, draw, fill=blue!20, 
    text width=5em, text centered, rounded corners, minimum height=4em]
\tikzstyle{line} = [draw, -latex']
\tikzstyle{output} = [trapezium,draw, trapezium left angle=70,trapezium right angle=-70,fill=pink, text width=3em, minimum height=2em,text centered]
\tikzstyle{data} = [trapezium,draw, trapezium left angle=70,trapezium right angle=-70,fill=white!20, text width=3em, minimum height=2em,text centered]
\tikzstyle{space} = [rectangle, fill=white,opacity=0,text width=5em, text centered, rounded corners, minimum height=4em]
\tikzstyle{results} = [ellipse,draw,fill=white, text width=4em, minimum height=3em,text centered]

%To generate list of symbols
\newcommand{\addsymbol}[3]{%
  \symboldisplay{#1}{#2}\\%
  \setelem{#3}{#1}
}
\newcommand{\symboldisplay}[2]{%
  $#1$ \parbox{5in}{\dotfill #2}%
  %$#1$ \parbox{5in}{ #2}%
}
%\def\setelem#1{\expandafter\def\csname myarray(#1)\endcsname}
\def\setelem#1{\expandafter\gdef\csname myarray(#1)\endcsname}
\def\dispsymbol#1{\csname myarray(#1)\endcsname} 

\usepackage{enumerate}




\usepackage{amsfonts}
\usepackage{tabularx,booktabs}

\newcolumntype{C}{>{\centering\arraybackslash}X} % centered version of "X" type
\setlength{\extrarowheight}{1pt}

 \newcolumntype{b}{>{\centering\arraybackslash\hsize=2.3\hsize}X}
\newcolumntype{s}{>{\centering\arraybackslash\hsize=.45\hsize}X}
\newcolumntype{m}{>{\centering\arraybackslash\hsize=.9\hsize}X}

\newcommand*{\Scale}[2][4]{\scalebox{#1}{$#2$}}%
\setcounter{MaxMatrixCols}{20}






% graphicx was written by David Carlisle and Sebastian Rahtz. It is
% required if you want graphics, photos, etc. graphicx.sty is already
% installed on most LaTeX systems. The latest version and documentation can
% be obtained at:
% http://www.ctan.org/tex-archive/macros/latex/required/graphics/
% Another good source of documentation is "Using Imported Graphics in
% LaTeX2e" by Keith Reckdahl which can be found as epslatex.ps or
% epslatex.pdf at: http://www.ctan.org/tex-archive/info/
%
% latex, and pdflatex in dvi mode, support graphics in encapsulated
% postscript (.eps) format. pdflatex in pdf mode supports graphics
% in .pdf, .jpeg, .png and .mps (metapost) formats. Users should ensure
% that all non-photo figures use a vector format (.eps, .pdf, .mps) and
% not a bitmapped formats (.jpeg, .png). IEEE frowns on bitmapped formats
% which can result in "jaggedy"/blurry rendering of lines and letters as
% well as large increases in file sizes.
%
% You can find documentation about the pdfTeX application at:
% http://www.tug.org/applications/pdftex

% *** MATH PACKAGES ***
%
%\usepackage[numbers]{natbib}
\usepackage{rotating}

\usepackage{nomencl}
\usepackage{tabularx,booktabs}
\newcolumntype{C}{>{\centering\arraybackslash}X} % centered version of "X" type
\setlength{\extrarowheight}{1pt}
\usepackage{lipsum}

%usepackage[hidelinks]{hyperref}
%for columns spanning multiple rows in tables
\usepackage{multirow}
%use the booktabs package to get (much!) better vertical spacing above and below "rules" (horizontal lines), resulting in a much more professional look of your tables.
%use the colortbl package to add color to tables.
\usepackage{booktabs,colortbl}
\usepackage{mathtools}
\DeclarePairedDelimiter{\abs}{\lvert}{\rvert}
% correct bad hyphenation here
%\hyphenation{op-tical net-works semi-conduc-tor}

  \renewcommand\footnoterule{\vspace*{-3pt}%
     \hrule width 2in height 0.4pt
     \vspace*{2.6pt}}
\usepackage[shortlabels]{enumitem}

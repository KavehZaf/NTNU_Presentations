\documentclass{beamer}


%%%%%%%%%%%%%%%%%%%%%
%%%%%%%%%%%  TiTle

%%%%%%%%%%%%%%%%%%%%%%%%%%%%%%
%%%%%%%%%%%%%%%%%%%%%%%%%%%%%%%%%

\usepackage[titletoc,toc,page]{appendix}
\renewcommand{\restoreapp}{}

\usepackage{ragged2e}




\usepackage[T1]{fontenc}
\usepackage[utf8]{inputenc}
\usepackage[english]{babel}






\setlength{\parindent}{4em}
\setlength{\parskip}{1em}
%\renewcommand{\baselinestretch}{2.0}

\usepackage{fourier}
\usepackage{booktabs}
\usepackage{threeparttablex}
\usepackage{longtable} 
\usepackage{lscape} 


\makeatletter
\newcommand\notsotiny{\@setfontsize\notsotiny\@vipt\@viipt}
\makeatother

\makeatletter
\newcommand\footnoteref[1]{\protected@xdef\@thefnmark{\ref{#1}}\@footnotemark}
\makeatother

\usepackage{threeparttable}

\usepackage{hyperref} 
\hypersetup{colorlinks=true,                
    breaklinks=true,                
    urlcolor= blue,                
    linkcolor= blue,                
    bookmarksopen=false,
    filecolor=black,
    citecolor=green,
    linkbordercolor=blue
}
\urlstyle{same}

\usepackage{amsmath} %equations and spacings
\usepackage{eurosym}
\usepackage{lipsum} % acces to random text
%graphics
\usepackage{graphicx}
%\usepackage[demo]{graphicx}
%caption styles
\usepackage[small,bf]{caption}
%\usepackage{caption}
\usepackage[labelformat=simple]{subcaption}
\renewcommand\thesubfigure{(\alph{subfigure})} % see subcaption doc
%<<byman 
\usepackage{colortbl}
\usepackage{xcolor}
\usepackage{color}
\usepackage{sidecap}
%byman>>

%<<byman
%% We want UK-english hyphenation patterns.byman
%\usepackage[british]{babel}
%byman>>

%Remove Paragraph indenting
\setlength{\parindent}{0.0in} 
\setlength{\parskip}{0.1in}

%<<byman
%% Use scalable, PostScript Type 1 versions of the Computer Modern fonts.
\usepackage{type1cm}
%% Replace the standard Computer Modern Typewriter font LaTeX uses
%% for monospace text with the PostScript font Adobe Courier.
\usepackage{courier}
\usepackage[T1]{fontenc}
\usepackage{ae,aecompl}
\usepackage{times}
%% Redefine the font used for the section headings to
%% Helvetica-Narrow Bold.
% for colour shades in tables 
\usepackage{colortbl} 
 %to control the title formating and spaces around the titles
 \usepackage{pdfpages}
\usepackage[compact]{titlesec}
%%byman>>

%spaces under sections
\usepackage[compact]{titlesec}
\titlespacing{\section}{0pt}{*0}{*0}
\titlespacing{\subsection}{0pt}{*0}{*0}
\titlespacing{\subsubsection}{0pt}{*0}{*0}
%fonts
 %\usepackage[T1]{fontenc} %gives light font
 %\usepackage[light,math]{iwona}

%Abbreviations or Acronyms
%\usepackage[intoc]{nomencl}
\usepackage{nomencl}
\renewcommand{\nomname}{List of Abbreviations}
\makenomenclature

\usepackage{csquotes}


%%%%%%%%%%%%%%%%%%%%%%%%%%%%%%%%%%%%%%%%%%%%
%%%%%%%%%%%%%%%%%%%%%%%%%%%%%%%%%%%%%%%%%%%%
%% Import from Sigurd Thesis




%Bibliography
%\usepackage{url}
%\usepackage{natbib}%\usepackage[sectionbib]{natbib}
%\usepackage{chapterbib}
%\bibpunct[:]{(}{)}{;}{a}{}{,} %citation structure
%\bibpunct{(}{)}{,}{a}{}{;} 
%\bibpunct{[}{]}{,}{a}{}{;}
%\bibpunct{(}{)}{;}{a}{}{,} % to follow the A&A style
%\usepackage{chapterbib}
%\usepackage{hyperref}
%\hypersetup{colorlinks=true,citecolor=blue}
%\hypersetup{colorlinks=true,citecolor=blue,linkcolor=blue,urlcolor=blue}

%To change box color around the links and citations, you have these other options :
%\hypersetup{citebordercolor=Violet,filebordercolor=Red,linkbordercolor=Blue}

\newcommand{\cmt}[1]{}%inline comment
  
%tabularx
\usepackage{tabularx}
\usepackage{tabulary}
\usepackage{multirow,longtable} %table on several pages
\usepackage{booktabs}

%enumerate(control spacing)
\newenvironment{packed_enum}{
\begin{enumerate}
	\vspace*{-1}
  	\setlength{\itemsep}{1pt}
  	\setlength{\parskip}{0pt}
  	\setlength{\parsep}{0pt}
}{\end{enumerate}}

%<<byman
\usepackage{multirow}
\usepackage{dcolumn}
\newcolumntype{d}{D{.}{.}{-2}}
\newcommand{\tabincell}[1]{\begin{tabular}{c}#1\end{tabular}}
\newcommand{\tabincellt}[1]{\begin{tabular}{l}#1\end{tabular}}
\newcommand{\merge}[1]{\multicolumn{1}{c}{#1}}

\usepackage{array}
\newcommand{\PreserveBackslash}[1]{\let\temp=\\#1\let\\=\temp}
\newcolumntype{C}[1]{>{\PreserveBackslash\centering}p{#1}}
\newcolumntype{R}[1]{>{\PreserveBackslash\raggedleft}p{#1}}
\newcolumntype{L}[1]{>{\PreserveBackslash\raggedright}p{#1}}

% Used to create tables with rows/cols spanning over se
%% Use fancy chapter headers, with Jos Dingjan's modifications,
%% plus my own tweaks. This style is not part of teTeX,
%% so we are using a local (and renamed) copy.
%\usepackage[Lenny]{fncychapleo}
\usepackage{fncychap}

%% Nicely format and linebreak URLs in the bibliography (and elsewhere).
%%\usepackage{url}
%% Define a new 'leo' style for the package that will use a smaller font.
%\makeatletter
%\def\url@leostyle{%
%  \@ifundefined{selectfont}{\def\UrlFont{\sf}}{\def\UrlFont{\small\ttfamily}}}
%\makeatother
%% Now actually use the newly defined style.
%\urlstyle{leo}
%byman>>

%graphics
\usepackage{epstopdf} %support for eps.
\DeclareGraphicsExtensions{.eps,.ps,.pdf,.png,.jpg}
\usepackage{float} % figure placing [H]

%landscape Option
\usepackage{lscape} %Left down
%usepackage{pdflscape} %Left up)

\usepackage[draft]{todonotes}   % notes showed (JJUNJU)
% Select what to do with command \comment:  
% \newcommand{\comment}[1]{}  %comment not showed
\newcommand{\comment}[1]
{\par {\bfseries \color{red} #1 \par}} %comment showed

%Section Numbering and TOC depth
\setcounter{secnumdepth}{2}
\setcounter{tocdepth}{3}

%chapter biblio
%\usepackage{chapterbib}

%Drawing flow charts
\usepackage{tikz}
\usetikzlibrary{shapes,arrows}
\tikzstyle{decision} = [diamond, draw, fill=blue!20, 
    text width=4.5em, text badly centered, node distance=3cm, inner sep=0pt]
\tikzstyle{process} = [rectangle, draw, fill=blue!20, 
    text width=5em, text centered, rounded corners, minimum height=4em]
\tikzstyle{line} = [draw, -latex']
\tikzstyle{output} = [trapezium,draw, trapezium left angle=70,trapezium right angle=-70,fill=pink, text width=3em, minimum height=2em,text centered]
\tikzstyle{data} = [trapezium,draw, trapezium left angle=70,trapezium right angle=-70,fill=white!20, text width=3em, minimum height=2em,text centered]
\tikzstyle{space} = [rectangle, fill=white,opacity=0,text width=5em, text centered, rounded corners, minimum height=4em]
\tikzstyle{results} = [ellipse,draw,fill=white, text width=4em, minimum height=3em,text centered]

%To generate list of symbols
\newcommand{\addsymbol}[3]{%
  \symboldisplay{#1}{#2}\\%
  \setelem{#3}{#1}
}
\newcommand{\symboldisplay}[2]{%
  $#1$ \parbox{5in}{\dotfill #2}%
  %$#1$ \parbox{5in}{ #2}%
}
%\def\setelem#1{\expandafter\def\csname myarray(#1)\endcsname}
\def\setelem#1{\expandafter\gdef\csname myarray(#1)\endcsname}
\def\dispsymbol#1{\csname myarray(#1)\endcsname} 

\usepackage{enumerate}




\usepackage{amsfonts}
\usepackage{tabularx,booktabs}

\newcolumntype{C}{>{\centering\arraybackslash}X} % centered version of "X" type
\setlength{\extrarowheight}{1pt}

 \newcolumntype{b}{>{\centering\arraybackslash\hsize=2.3\hsize}X}
\newcolumntype{s}{>{\centering\arraybackslash\hsize=.45\hsize}X}
\newcolumntype{m}{>{\centering\arraybackslash\hsize=.9\hsize}X}

\newcommand*{\Scale}[2][4]{\scalebox{#1}{$#2$}}%
\setcounter{MaxMatrixCols}{20}






% graphicx was written by David Carlisle and Sebastian Rahtz. It is
% required if you want graphics, photos, etc. graphicx.sty is already
% installed on most LaTeX systems. The latest version and documentation can
% be obtained at:
% http://www.ctan.org/tex-archive/macros/latex/required/graphics/
% Another good source of documentation is "Using Imported Graphics in
% LaTeX2e" by Keith Reckdahl which can be found as epslatex.ps or
% epslatex.pdf at: http://www.ctan.org/tex-archive/info/
%
% latex, and pdflatex in dvi mode, support graphics in encapsulated
% postscript (.eps) format. pdflatex in pdf mode supports graphics
% in .pdf, .jpeg, .png and .mps (metapost) formats. Users should ensure
% that all non-photo figures use a vector format (.eps, .pdf, .mps) and
% not a bitmapped formats (.jpeg, .png). IEEE frowns on bitmapped formats
% which can result in "jaggedy"/blurry rendering of lines and letters as
% well as large increases in file sizes.
%
% You can find documentation about the pdfTeX application at:
% http://www.tug.org/applications/pdftex

% *** MATH PACKAGES ***
%
%\usepackage[numbers]{natbib}
\usepackage{rotating}

\usepackage{nomencl}
\usepackage{tabularx,booktabs}
\newcolumntype{C}{>{\centering\arraybackslash}X} % centered version of "X" type
\setlength{\extrarowheight}{1pt}
\usepackage{lipsum}

%usepackage[hidelinks]{hyperref}
%for columns spanning multiple rows in tables
\usepackage{multirow}
%use the booktabs package to get (much!) better vertical spacing above and below "rules" (horizontal lines), resulting in a much more professional look of your tables.
%use the colortbl package to add color to tables.
\usepackage{booktabs,colortbl}
\usepackage{mathtools}
\DeclarePairedDelimiter{\abs}{\lvert}{\rvert}
% correct bad hyphenation here
%\hyphenation{op-tical net-works semi-conduc-tor}

  \renewcommand\footnoterule{\vspace*{-3pt}%
     \hrule width 2in height 0.4pt
     \vspace*{2.6pt}}
\usepackage[shortlabels]{enumitem}

\title{\large \textbf{Large-Scale Integration of EVs, one Application of the High-Performace Solver: BATTPOWER}}
\begin{backgroundblock}{20mm}{20mm}
    \begin{tikzpicture}
    \node[anchor=east,inner sep=0] (B) at (4,0) {\includegraphics[width=\linewidth]{Figures/PowerSys.png}};
           \fill [draw=none, fill=white, fill opacity=0.7] (B.north west) -- (B.north east) -- (B.south east) -- (B.south west) -- (B.north west) -- cycle;
 
    \end{tikzpicture} 
    \end{backgroundblock} 

\logo{%
    \includegraphics[width=3cm,height=3cm,keepaspectratio]{ntnulogo_eng}~%
}


\author{Salman Zaferanlouei}
\institute{NTNU{\\ \vskip 1cm} \scalebox{1.5}{\insertlogo}}
\date{\today}









\begin{document}
\begin{frame}
\titlepage
\end{frame}
%%%%%%%%%%%%%%%%%%%%%%%%%%%%%%%%%%%%%%%%%%
%%%%%%%%%%%%%%%%%%%%%%%%%%%%%%%%%%%%%%%%%%
%%%%%%%%%%%%%%%%%%%%%%%%%%%%%%%%%%%%%%%%%%
%%%%%%%%%%%%%%%%%%%%%%%%%%%%%%%%%%%%%%%%%%
%%%%%%%%%%%%%%%%%%%%%%%%%%%%%%%%%%%%%%%%%
%%%%%%%%%%%%%%%%%%%%%%%%%%%%%%%%%%%%%%%%%%
%%%%%%%%%%%%%%%%%%%%%%%%%%%%%%%%%%%%%%%%%%
%%%%%%%%%%%%%%%%%%%%%%%%%%%%%%%%%%%%%%%%%%





\section{Purpose/Content}
\begin{frame}{The presentation goal}
\begin{itemize}
\item \textbf{Purpose:} To discuss how to solve multi-period optimal power flow fast; \textcolor{red}{\textbf{Large-Scale simulation with respect to time and space.}}
\item \textbf{Phase I:} Background and Motivation
\item \textbf{Phase II:} Power Flow and Optimal Power Flow
\item \textbf{Phase III:} Solution Method
\item \textbf{Phase IV:} Speed up
\item \textbf{Phase V:} Future Work

\end{itemize}
\begin{center}
\begin{tabular}{|l l|} 
\hline
\textbf{Presentation Time:}& 15-20 min \\
\hline
\end{tabular}
\end{center}
\end{frame}
%%%%%%%%%%%%%%%%%%%%%%%%%%%%%%%%%%%%%%%%%%
%%%%%%%%%%%%%%%%%%%%%%%%%%%%%%%%%%%%%%%%%%


\begin{frame}[plain]
\begin{tikzpicture}[overlay, remember picture]
\node[anchor=center] at (current page.center) {
\begin{beamercolorbox}[center]{title}
     Phase I:\\\textbf{Background and Motivation}
  \end{beamercolorbox}};
\end{tikzpicture}

\end{frame}
%%%%%%%%%%%%%%%%%%%%%%%%%%%%%%%%%%%%%%%%%
%%%%%%%%%%%%%%%%%%%%%%%%%%%%%%%%%%%%%%%%%%
%%%%%%%%%%%%%%%%%%%%%%%%%%%%%%%%%%%%%%%%%%
%%%%%%%%%%%%%%%%%%%%%%%%%%%%%%%%%%%%%%%%%
%%%%%%%%%%%%%%%%%%%%%%%%%%%%%%%%%%%%%%%%%%
%%%%%%%%%%%%%%%%%%%%%%%%%%%%%%%%%%%%%%%%%%
%%%%%%%%%%%%%%%%%%%%%%%%%%%%%%%%%%%%%%%%%%
%%%%%%%%%%%%%%%%%%%%%%%%%%%%%%%%%%%%%%%%%%
\section{Background}
\begin{frame}{IBM 7090}
\begin{figure}[!htbp]
\centering
\includegraphics[width=4 in , height=2.4 in]{Figures/ibm7090nasa.jpg}
\caption{\tiny[Photo: NASA ]}
\end{figure}
\end{frame}


\begin{frame}
\begin{block}{Items}
\begin{itemize}
\item <1-> \small It took up a whole room and it was state of the art when it came out in 1960!
\item<2-> \small  It costs 2.9 million dollars u.s. at the time which is a lot more now!
\item<3->\small It could perform a hunderd thousand floating point operation per second (we call it flops)
\only<4>{ \begin{alertblock}{\emoji{thinking-face}}
{\small So Does that sound fast? pretty fast?}
\end{alertblock}}
\only <5-> { \begin{alertblock}{\emoji{astonished-face}}
{\small a first generation Raspberry Pi; The little thing that everybody builds their toy projects with these days was over 40 times faster!}
\end{alertblock}}
\end{itemize}
\end{block}
\begin{figure}
\includegraphics[scale=0.2]{Figures/ibm7090nasa.jpg}
 \caption{NASA}
\end{figure}
\end {frame}


\begin{frame}
\begin{block}{How things got better?}
\onslide <1-> Make Bigger, Faster computers.
\end{block}
For a long time we just basically changed the hardware to make computers better by stuffing more transistors onto a chip and running instruction through a system more quickly. We built bigger and faster computers as they get cheaper, we can through more of them in a given problem.
\end {frame}

\begin{frame}{CPU}
Gordon Moore has quantified this growth in 1965 when he coined Moore's law! Moore's law says that the number of components on a chip would double every 18 months.\\
Doubling number of components every 18 months meant the computers got twice as powerful while sitting in the same space.
\begin{figure}[!htbp]
\centering
\includegraphics[width=3 in , height=2 in]{Figures/CPU.jpg}
\caption{\tiny[Image bia pixabay ]}
\end{figure}
\end{frame}

\begin{frame}{Problems/arguments}
However there is a few problems with this:
\begin{enumerate}[i.]
\item <1->{\tiny The law of Physics says that eventually you can't make a component any smaller than a certain size,}
\item <2->  {\tiny It gets more expensive to build smaller chips as you get close to those limits.}
\item <3-> {\tiny But customers, the people who use and buy computers and keep the computer manufacurers in business do not care!}
\item <4-> {\tiny Programmers keep writing bigger and more complex programs and as long as the new computers are speeding up more rapidly than your software is growing, things seem to be improving!}
\end{enumerate}

\begin{figure}[!htbp]
\centering
\includegraphics[width=3 in , height=2 in]{Figures/oldcomp.jpg}
\caption{\tiny[flickr ]}
\end{figure}
\end{frame}


\begin{frame}{intr.}
When a program needs a bigger and newer computer to run it what happens to the old one? It might handed down to someone or resold or it might get thrown out,\\
And this is not good for the planet!\\
Also, having code only works well on the newest devices is not great for the people who can not afford the sate of the art!\\
If you have a business, you have a problem, because you have to get your code or your website on whatever people are using, rather than what you wish they were using, So if your program is so slow, they might not be your custormer for a very long,\\
\begin{figure}[!htbp]
\centering
\includegraphics[width=3 in , height=2 in]{Figures/laptop.jpg}
\caption{\tiny[Getty images ]}
\end{figure}
\end{frame}

\begin{frame}{Intro}
People want their computers keep getting better But, Just making the hardware faster is not nearly enough anymore!\\
And Throwing more hardware at a problem can get VERY expensive.\\
It seems crutial to modify softwares at the same time.
\end{frame}


%%%%%%%%%%%%%%%%%%%%%%%%%%%%%%%%%%%%%%%%%
%%%%%%%%%%%%%%%%%%%%%%%%%%%%%%%%%%%%%%%%%%
%%%%%%%%%%%%%%%%%%%%%%%%%%%%%%%%%%%%%%%%%%
%%%%%%%%%%%%%%%%%%%%%%%%%%%%%%%%%%%%%%%%%
%%%%%%%%%%%%%%%%%%%%%%%%%%%%%%%%%%%%%%%%%%
%%%%%%%%%%%%%%%%%%%%%%%%%%%%%%%%%%%%%%%%%%
%%%%%%%%%%%%%%%%%%%%%%%%%%%%%%%%%%%%%%%%%%
%%%%%%%%%%%
\begin{frame}{Electricity Grid}
\begin{figure}[!htbp]
\centering
\includegraphics[width=2.8 in , height=2.4 in]{Figures/EVchalendge1.png}
\caption{\tiny[www.sciencedirect.com/science/article/pii/B9781845697846500019 ]}
\end{figure}
\end{frame}


%
%%%%%%%%%%%%%%%%%%%%%%%%%%%%%%%%%%%%%%%%%%
%%%%%%%%%%%%%%%%%%%%%%%%%%%%%%%%%%%%%
%%%%%%%%%%%%%%%%%%%%%%%%%%%%%%%%%%%%%%%%%%
%%%%%%%%%%%%%%%%%%%%%%%%%%%%%%%%%%%%%%%%%%
%%%%%%%%%%%%%%%%%%%%%%%%%%%%%%%%%%%%%%%%%%
%%%%%%%%%%%%%%%%%%%%%%%%%%%%%%%%%%%%%%%%%%
%\section{Background}
%\begin{frame}{Electricity Grid}
%\begin{figure}[!htbp]
%\centering
%\includegraphics[width=2.8 in , height=2.4 in]{Figures/EVchalendge1.png}
%\caption{\tiny[www.sciencedirect.com/science/article/pii/B9781845697846500019]}
%\end{figure}
%\end{frame}

%%%%%%%%%%%%%%%%%%%%%%%%%%%%%%%%%%%%%%%%%%
%%%%%%%%%%%%%%%%%%%%%%%%%%%%%%%%%%%%%%%%%%
%%%%%%%%%%%%%%%%%%%%%%%%%%%%%%%%%%%%%%%%%%
%%%%%%%%%%%%%%%%%%%%%%%%%%%%%%%%%%%%%%%%%%

\begin{frame}{Sustainability/Green shift/CO\textsubscript{2} reduction}
\begin{alertblock}{\textcolor{red}{For many reasons power electricity grid is facing decentralization}}
\begin{itemize}
\item<1-> Phasing out coal (carbon-heavy sources of production) and nuclear power plants
\item<2-> Increase penetration of solar and wind production
\end{itemize}
\end{alertblock}
\begin{columns}
    \column{0.5\textwidth}
    This chain between large power producers and consumers is weakened.
    \column{0.5\textwidth}
\includegraphics[width=2 in , height=1.6 in]{Figures/EVchalendgebreak.png}
\end{columns}
\end{frame}

%%%%%%%%%%%%%%%%%%%%%%%%%%%%%%%%%%%%%%%%%%
%%%%%%%%%%%%%%%%%%%%%%%%%%%%%%%%%%%%%%%%%%
%%%%%%%%%%%%%%%%%%%%%%%%%%%%%%%%%%%%%%%%%%
%%%%%%%%%%%%%%%%%%%%%%%%%%%%%%%%%%%%%%%%%%
\section{Background}
\begin{frame}{Background: cont.}
\begin{block}{Items}
\begin{itemize}
\item <1-> \small Green shift in electricity systems is needed for the reduction of $\mathrm{CO_2}$ emissions
\begin{itemize}  
\item<2-> \tiny  Integration of Distributed Energy Resources (DER) is a huge challenge
\end{itemize}
\only<2>{ \begin{alertblock}{\tiny Note}
{\tiny DER includes Renewable Energy, Energy Storage, Electric Vehicles and Flexible Demand}
\end{alertblock}}
\item<3->\small Grid companies must be able to analyse the impacts of DER
\only<4>{ \begin{alertblock}{Note}
\textbf{\tiny Optimal Power Flow (OPF)}{\tiny solvers are essential}
\end{alertblock}}
\item<5->\small The inegration of DER in smart grids calls for \textbf{much more sophisticated solvers} for OPF
\end{itemize}
\end{block}
\begin{figure}
\includegraphics[scale=0.06]{Figures/PowerSys.png}
 \caption{A Typical Power System \textcolor{gray}{\tiny[Rochester Gas \& Electricity]}}
\end{figure}

\end{frame}
        %%%%%%%%%%%%%%%%%%%%%%%%%%%%%%%%%
%%%%%%%%%%%%%%%%%%%%%%%%%%%%%%%%%%%
%%%%%%%%%%%%%%%%%%%%%%%%%%%%%%%%%%%%
%%%%%%%%%%%%%%%%%%%%%%%%%%%%%%%%%%%
%%%%%%%%%%%%%%%%%%%%%%%%%%%%%%%%%
\begin{frame}{Challenges in the planning and operation of the grid}
\begin{itemize}
\item<1-> \textbf{Planning:} Optimizing the right type, size and timing of new grid investments
\begin{itemize}
\item Local generation (e.g. PV) and increased load (e.g. EVs)  can be located in areas where the grid is weak
\item Energy storage and demand flexibility are alternatives to grid reinforcements
\end{itemize}

\item<2-> \textbf{Operation:}Optimize the use of controllable assets such as energy storage and flexible demand to secure, reliable and economic operation of the distribution grids. This means:
\begin{itemize}
\item Making the right use of Demand Response
\item being able to value the use of end-user flexibility for local or system-wide grid services
\item Simulating and optimizing the grids in the presence of \textbf{future local markets for energy and flexibility}
\end{itemize}
\end{itemize}
\end{frame}
%%%%%%%%%%%%%%%%%%%%%%%%%%%%%%%%%
%%%%%%%%%%%%%%%%%%%%%%%%%%%%%%%%%%%
%%%%%%%%%%%%%%%%%%%%%%%%%%%%%%%%%%%%
%%%%%%%%%%%%%%%%%%%%%%%%%%%%%%%%%%%
%%%%%%%%%%%%%%%%%%%%%%%%%%%%%%%%%

\begin{frame}{Limitations of traditional grid operation and planning}
\vskip -1cm
\begin{block}{Notes}
{\scriptsize
\begin{itemize}
\item<1-> Classical single-period OPF does not offer a possibility for optimal operational scheduling of storage and flexible demand
\item<2->  We therefore aim to develop the foundations for a new generation of Multi-Period OPF (MPOPF) solvers
\begin{enumerate}[i.]
{\tiny
\item Solves the OPF problem over several coupled time-steps
\item Computation time is an issue when using both commercial or free optimization solvers
}
\end{enumerate}
\item<3-> MPOPF is an extremely challenging scientific task:
\begin{enumerate}[i.]
{\tiny
\item Nonlinearity
\item Large-scale problem with respect with to time and space
\item Involves stochastic generations and load}
\end{enumerate}
\end{itemize}}
\onslide<4>{\begin{alertblock}{Hardware is reaching its limit with respect to CPU clock speed}
\end{alertblock}}
\end{block}
\begin{backgroundblock}{10mm}{50mm}
 \begin{tikzpicture}
    \node[anchor=east,inner sep=0] (B) at (4,0) {\includegraphics[scale=0.8]{Figures/CPUSpeed.jpg}};
           \fill [draw=none, fill=white, fill opacity=0.5] (B.north west) -- (B.north east) -- (B.south east) -- (B.south west) -- (B.north west) -- cycle;
    \end{tikzpicture}
   \end{backgroundblock}
\end{frame}
%%%%%%%%%%%%%%%%%%%%%%%%
%%%%%%%%%%%%%%%%%%%%%%%%%%%%%%%%
        %%%%%%%%%%%%%%%%%%%%%%%%%%%%%%%%%
%%%%%%%%%%%%%%%%%%%%%%%%%%%%%%%%%%%
%%%%%%%%%%%%%%%%%%%%%%%%%%%%%%%%%%%%
%%%%%%%%%%%%%%%%%%%%%%%%%%%%%%%%%%%
%%%%%%%%%%%%%%%%%%%%%%%%%%%%%%%%%
\begin{frame}{Solution:}
\begin{block}{High-Performance Solver [1-2]}
\begin{itemize}
\item<1-> Algorithmic design tailored to the conventional OPF algorithms speed-up the solution proposal
\item<1-> Prototype model shows convincing results for real-sized system with distributed renewables, storages and EVs

\begin{enumerate}[i.]
{\tiny
\item A high-performance and memory-efficient sparse algorithm
\item Utilizing the structure of the underlying mathematical formulation}
\end{enumerate}
\end{itemize}
\end{block}

\onslide<2->{\begin{block}{Benefits}
\begin{itemize}
\item<2-> Optimal utilization of \textbf{stored energy} and \textbf{flexibility} where and when it creates the highest value for the system
\item<3-> Can be used for grid planning, grid operation and local markets
\end{itemize}
\end{block}}
\begin{enumerate}
{\tiny
\item S. Zaferanlouei, H. Farahmand, V. V. Vadlamudi, M. Korpås,“BATTPOWER Toolbox: Memory-Efficient and High-Performance Multi-Period AC Optimal Power Flow Solver”, IEEE Transactions on Power Systems, Jan. 16th, 2021.
\item S. Zaferanlouei, et al., “BATTPOWER Application: Large-Scale Integration of EVs in an Active Distribution Grid ---A Norwegian Case Study”, Under review in the journal of EPSR}
 \end{enumerate}
\end{frame}
%%%%%%%%%%%%%%%%%%%%%%%%%%%%%%%%%%%%%%%%%
%%%%%%%%%%%%%%%%%%%%%%%%%%%%%%%%%%%%%%%%%




%%%%%%%%%%%%%%%%%%%%%%%%%%%%%%%
\begin{frame}{Power System--- Today}
\centering
\animategraphics[loop,width=10cm]{12}{Figures/gif/Today/01-}{0}{115}
\end{frame}
%%%%%%%%%%%%%%%%%%%%%%%%%%%%%%%%%%%%%%%%%%
%%%%%%%%%%%%%%%%%%%%%%%%%%%%%%%%%%%%%%%%%%
%%%%%%%%%%%%%%%%%%%%%%%%%%%%%%%%%%%%%%%%%%
%%%%%%%%%%%%%%%%%%%%%%%%%%%%%%%%%%%%%%%%%%

\begin{frame}{Power System--- Future}
\centering
\animategraphics[loop,width=10cm]{12}{Figures/gif/Future/01-}{0}{98}
\end{frame}

%%%%%%%%%%%%%%%%%%%%%%%%%%%%%%%%%%%%%%%%%%
%%%%%%%%%%%%%%%%%%%%%%%%%%%%%%%%%%%%%%%%%%
%%%%%%%%%%%%%%%%%%%%%%%%%%%%%%%%%%%%%%%%%%
%%%%%%%%%%%%%%%%%%%%%%%%%%%%%%%%%%%%%%%%%%


\begin{frame}[plain]
\begin{tikzpicture}[overlay, remember picture]
\node[anchor=center] at (current page.center) {
\begin{beamercolorbox}[center]{title}
     Phase II:\\\textbf{Power Flow and Optimal Power Flow}
  \end{beamercolorbox}};
\end{tikzpicture}

\end{frame}

%%%%%%%%%%%%%%%%%%%%%%%%%%%%%%%%% 
%%%%%%%%%%%%%%%%%%%%%%%%%%%%%%%%%%% 
%%%%%%%%%%%%%%%%%%%%%%%%%%%%%%%%%%%% 
%%%%%%%%%%%%%%%%%%%%%%%%%%%%%%%%%%% 
%%%%%%%%%%%%%%%%%%%%%%%%%%%%%%%%% 
\section{Power Flow} 
\begin{frame}{Power Flow Equations} 
\vskip -0.4cm 
 
\begin{block}{Source of Non-linearity} 
\textcolor{red}{Nonlinear relationship} between the voltage phasors and the power injections 
\end{block} 
\vskip -0.5cm 
\begin{table}[htbp!] 
\begin{tabular}{|r l| l|}  
\hline 
\multicolumn{3}{|c|}{ \includegraphics[scale=.15]{Figures/PowerFlow.png}}\\ 
\hline 
 $\textcolor{blue}{P_i}+j\textcolor{blue}{Q_i} $&$= \textcolor{green}{V_i}.\overline{I}_i$& \\ 
 &$=\textcolor{green}{V_i}.(\mathbf{\overline{Y}}_i.\textcolor{green}{\overline{V}})$&{\tiny $\mathbf{{Y}}$ \ admittance matrix}\\ 
 &$=\textcolor{green}{V_i}. (\mathbf{G}_i-j\mathbf{B}_i)\textcolor{green}{\overline{V}}$&{\tiny $\mathbf{{G}}$ and $\mathbf{{B}}$ \  conductance and susceptance matrices}\\ 
 $\textcolor{green}{|V_i|^2}$&$=\textcolor{green}{V_i}.\textcolor{green}{\overline{V}_i}$&{\tiny $\mathbf{{Y}}=\mathbf{G}+j\mathbf{B}$}\\ 
 \multicolumn{2}{|c|}{"Power Flow Equations"}&\\ 
 \multicolumn{2}{|c|}{"Load Flow Equations"}&\\ 
 \hline 
 \end{tabular} 
\end{table} 
\begin{itemize}[label={>}] 
\item \textcolor{red}{Coupled quadratics} in complex voltage phasors 
\end{itemize} 
\end{frame} 
 
%%%%%%%%%%%%%%%%%%%%%%%%%%%%%%%%% 
%%%%%%%%%%%%%%%%%%%%%%%%%%%%%%%%%%% 
%%%%%%%%%%%%%%%%%%%%%%%%%%%%%%%%%%%% 
%%%%%%%%%%%%%%%%%%%%%%%%%%%%%%%%%%%
%%%%%%%%%%%%%%%%%%%%%%%%%%%%%%%%%
\begin{frame}{Power Flow Equations in Different Coordinates}

\onslide<1-2>{\scriptsize {The \textcolor{red}{bus injection} model
\begin{center}
\begin{tabular}{|c c|}
\hline
\multicolumn{2}{|c|}{ \includegraphics[scale=.15]{Figures/PowerFlow.png}}\\
\hline
 \multicolumn{1}{|c|}{Polar Voltage Coordinates}&$\textcolor {green}{V_i}=\textcolor {green}{\abs{V_i}}\angle \textcolor {green}{\theta_i}$\quad $\textcolor {green}{\theta_1}=0$\\
\cline{1-1}
\multicolumn{2}{|c|}{$ 
 \textcolor {blue}{P_{i}} =\textcolor {green}{\abs{V_{i}}}\sum_{j=1}^{N} \textcolor {green}{\abs{V_{j}}} (\boldsymbol{G}_{ij} cos(\textcolor {green}{\theta{i}}-\textcolor {green}{\theta{j}})+\boldsymbol{B}_{ij}sin(\textcolor {green}{\theta{i}}-\textcolor {green}{\theta{j}}))$}\\

\multicolumn{2}{|c|}{$ \textcolor {blue}{Q_{i}} =\textcolor {green}{\abs{V_{i}}}\sum_{j=1}^{N}
 \textcolor {green}{\abs{V_{j}}} (\boldsymbol{G}_{ij} sin(\textcolor {green}{\theta{i}}-\textcolor {green}{\theta{j}})-\boldsymbol{B}_{ij}cos(\textcolor {green}{\theta{i}}-\textcolor {green}{\theta{j}}))$}\\
 \hline\hline
  \multicolumn{1}{|c|}{Rectangular Voltage Coordinates}&$\textcolor {green}{V_i}=\textcolor {green}{V_{di}}+j \textcolor {green}{V_{qi}}$\quad $\textcolor {green}{V_{q1}}=0$\\
\cline{1-1}
\multicolumn{2}{|c|}{$ 
 \textcolor {blue}{P_{i}} =\textcolor {green}{V_{di}}(\boldsymbol{G}_{ij} \textcolor {green}{V_{dj}} -\boldsymbol{B}_{ij} \textcolor {green}{V_{qj}})
 +
 \textcolor {green}{V_{qi}}(\boldsymbol{B}_{ij} \textcolor {green}{V_{dj}} +\boldsymbol{G}_{ij} \textcolor {green}{V_{qj}})$}\\

\multicolumn{2}{|c|}{$ 
 \textcolor {blue}{Q_{i}} =-\textcolor {green}{V_{di}}(\boldsymbol{B}_{ij} \textcolor {green}{V_{dj}} +\boldsymbol{G}_{ij} \textcolor {green}{V_{qj}})
 +
 \textcolor {green}{V_{qi}}(\boldsymbol{G}_{ij} \textcolor {green}{V_{dj}} -\boldsymbol{B}_{ij} \textcolor {green}{V_{qj}})$}\\

$ \textcolor {green}{V_{di}}= Re (\textcolor {green}{V_i})$ \  $ \textcolor {green}{V_{qi}}= Im (\textcolor {green}{V_i})$& $N=$ number of Nodes\\
  \hline
\end{tabular}
\end{center}}}
\onslide<2-> {\scriptsize \begin{block}{Power Flow}
PF equations are a set of nonlinear algebraic equations which can be solved with Newton-Raphson, Gauss–Seidel, Fast-decoupled-load-flow method and etc.
\end{block}}

\end{frame}

%%%%%%%%%%%%%%%%%%%%%%%%%%%%%%%%%%%%
%%%%%%%%%%%%%%%%%%%%%%%%%%%%%%%%%%%
%%%%%%%%%%%%%%%%%%%%%%%%%%%%%%%%%
\subsection{Single Period ACOPF}
\begin{frame}{Single period ACOPF - problem formulation}
\begin{block}{Single Period AC Optimal Power Flow (ACOPF)}
We are controlling some variables in OPF to minimise system costs with respect to constraints.
\end{block}
\begin{center}
\begin{tabular}{|l l|}
\hline
Objective Function&$\min_{\mathbf{x}} f(\mathbf{x})$\\
\rowcolor{Gray}
\begin{tabular}[l]{@{}l@{}}Equality Constraints\\ (Balance)\\ Power Flow \end{tabular}  &$\textrm{s.t. }  \mathbf{g}(\mathbf{x})= \begin{colorbmatrix}
   \mathbf{\widetilde{g}}(\mathbf{x})\\
   \mathbf{\overline{g}}(\mathbf{x})
\end{colorbmatrix}=0 \qquad \in   \mathbb{R}^{n_{gx} \times 1}$\\

\begin{tabular}[l]{@{}l@{}}Inequality constraints\\ (line and transformer)\\ Operational Constraints \end{tabular}  & $\mathbf{h}(\mathbf{x})=\begin{colorbmatrix}
   \mathbf{\widetilde{h}}(\mathbf{x})\\
   \mathbf{\overline{h}}(\mathbf{x})
\end{colorbmatrix}\leq 0 \qquad \in   \mathbb{R}^{n_{hx} \times 1} $\\
\rowcolor{Gray}
Vector of Variables&$ \mathbf{x}= \big[\boldsymbol{\Theta}\  \boldsymbol{\mathcal{V}} \  \boldsymbol{\mathcal{P}}^{\mathrm{g}} \ \boldsymbol{\mathcal{Q}}^{\mathrm{g}} \big]^\top
 \in \mathbb{R}^{n_{x}\times 1}$\\
\hline
\end{tabular}
\end{center}
\end{frame}
%%%%%%%%%%%%%%%%%%%%%%%%%%%%%%%%%
%%%%%%%%%%%%%%%%%%%%%%%%%%%%%%%%%%%
%%%%%%%%%%%%%%%%%%%%%%%%%%%%%%%%%%%%
\subsection{Limitations of PF and single period OPF}
\begin{frame}
\onslide<1->
{\footnotesize
\begin{block}{Limitation of Power Flow}
\begin{tabular}{|l|}
\hline
\rowcolor{Gray}
\tabitem PF is only a set of static equations which provides status of a system\\
 \rowcolor{Gray}\qquad for time: $t=t_s$\\
\rowcolor{Gray} \tabitem It does not include  generation constraints and operational constraints\\
\hline
\end{tabular}
\end{block}
}
\onslide<2->
{\footnotesize
\begin{block}{
Limitation of Single Period AC Optimal Power Flow}

\begin{tabular}{|l|}
\hline
\rowcolor{Gray} \quad \tabitem OPF is an optimisation problem which optimises status of a system\\
\rowcolor{Gray}\ \ \qquad for a SINGLE time: $t=t_s$.\\
\rowcolor{Gray} \quad \tabitem Although it includes the operational constraints for single time,\\
\rowcolor{Gray}  \qquad it will not include operation of DERs and generators over a time horizon.\\
\rowcolor{Gray} \quad \tabitem It is not capable of integrating storage devices and EVs. \\
\hline
\end{tabular}
}
\end{block}
\onslide<3->
\begin{alertblock}
{ Our suggested approach: multi-period ACOPF}
{ Solves the OPF problem over several time-steps at once useful formulation for systems with Energy Storage and Shiftable Loads (e.g. Evs)}
\end{alertblock}
\end{frame}
%%%%%%%%%%%%%%%%%%%%%%%%%%%%%%%%%
%%%%%%%%%%%%%%%%%%%%%%%%%%%%%%%%%%%
%%%%%%%%%%%%%%%%%%%%%%%%%%%%%%%%%%%%
%%%%%%%%%%%%%%%%%%%%%%%%%%%%%%%%%%%
%%%%%%%%%%%%%%%%%%%%%%%%%%%%%%%%%
\subsection{MultiPeriod ACOPF}
\begin{frame}
\onslide<1->{
\begin{block}{Advantages of MultiPeriod ACOPF}
\textcolor{red}{>} Integrating dependent time power system's components such as stationary ESS, EV, generators ramp rate, and so on.\\}
\onslide<2-> {\textcolor{red}{>} between 2\% to 5\% less costly solutions.
\only<2>{\begin{backgroundblock}{50mm}{45mm}
        \includegraphics[scale=0.33]{Figures/ReceedingHorizon.png}
        \end{backgroundblock}}
\end{block}}
\visible<3-> {\begin{block}{Disadvantage of MultiPeriod ACOPF}
The problem grows very large as the number of time-steps is increased, which may lead to an intractable solution.
\end{block}}
\end{frame}
%%%%%%%%%%%%%%%%%%%%%%%%%%%%%%%%%%%
%%%%%%%%%%%%%%%%%%%%%%%%%%%%%%%%%%%%
%%%%%%%%%%%%%%%%%%%%%%%%%%%%%%%%%%%
%%%%%%%%%%%%%%%%%%%%%%%%%%%%%%%%%
\begin{frame}{MultiPeriod ACOPF-problem formulation}
\begin{center}
\begin{tabular}{|l l|}
\hline
{\tiny Objective Function}&$\min_{\mathbf{X}} F(\mathbf{X})$\\
\rowcolor{Gray}
{\tiny \begin{tabular}[l]{@{}l@{}}Equality Constraints\\ (Balance)\\ Power Flow \end{tabular} } &$\textrm{s.t. } G(\mathbf{X})= \begin{colorbmatrix}
   \widetilde{G}(\mathbf{X}) \
   \overline{G}(\mathbf{X}) \
   \overline{G}^s(\mathbf{X})
\end{colorbmatrix}^\top=0  \in   \mathbb{R}^{N_{g} \times 1}$\\

{\tiny \begin{tabular}[l]{@{}l@{}}Inequality constraints\\ (line and transformer)\\ Operational Constraints \end{tabular} } & $H(\mathbf{X})=\begin{colorbmatrix}
   \widetilde{H}(\mathbf{X}) \quad
   \overline{H}(\mathbf{X})
\end{colorbmatrix}^\top\leq 0  \in   \mathbb{R}^{N_{h} \times 1} $\\
\rowcolor{Gray}
{\tiny Vectors of Constraints}&{\tiny \begin{tabular}[l]{@{}l@{}}$
   \widetilde{\mathbf{G}}(\mathbf{X})=
\begin{colorbmatrix}
    \widetilde{\mathbf{g}}(\mathbf{x}_{1}) \
    \widetilde{\mathbf{g}}(\mathbf{x}_{2}) \
    \dots \
    \widetilde{\mathbf{g}}(\mathbf{x}_{T}) 
\end{colorbmatrix}^\top$\\ $ 
   \overline{\mathbf{G}}(\mathbf{X})=
\begin{colorbmatrix}
    \overline{\mathbf{g}}(\mathbf{x}_{1}) \
    \overline{\mathbf{g}}(\mathbf{x}_{2}) \
    \dots \
    \overline{\mathbf{g}}(\mathbf{x}_{T}) \
\end{colorbmatrix}^\top $\\ $
  \overline{\mathbf{G}}^s(\mathbf{X})=
\begin{colorbmatrix}
    \overline{\mathbf{g}}^s(\boldsymbol{\tau}_1) \
    \overline{\mathbf{g}}^s(\boldsymbol{\tau}_2) \
    \dots \
    \overline{\mathbf{g}}^s(\boldsymbol{\tau}_T) 
\end{colorbmatrix}^\top  $\\ $
   \widetilde{\mathbf{H}}(\mathbf{X})=
\begin{colorbmatrix}
    \widetilde{\mathbf{h}}(\mathbf{x}_{1}) \
    \widetilde{\mathbf{h}}(\mathbf{x}_{2}) \
    \dots \
    \widetilde{\mathbf{h}}(\mathbf{x}_{T}) 
\end{colorbmatrix}^\top$\\ $
  \overline{\mathbf{H}}(\mathbf{X})= 
\begin{colorbmatrix}
    \overline{\mathbf{h}}(\mathbf{x}_{1}) \
    \overline{\mathbf{h}}(\mathbf{x}_{2}) \
    \dots \
    \overline{\mathbf{h}}(\mathbf{x}_{T}) 
\end{colorbmatrix}^\top $\end{tabular}  
}\\  
\hline
\end{tabular}
\end{center}



\end{frame}
%%%%%%%%%%%%%%%%%%%%%%%%%%%%%%%%%
%%%%%%%%%%%%%%%%%%%%%%%%%%%%%%%%%%%
%%%%%%%%%%%%%%%%%%%%%%%%%%%%%%%%%%%%
%%%%%%%%%%%%%%%%%%%%%%%%%%%%%%%%%%%
%%%%%%%%%%%%%%%%%%%%%%%%%%%%%%%%%

%%%%%%%%%%%%%%%%%%%
%%%%%%%%%%%%%%%
\begin{frame}[plain]
\begin{tikzpicture}[overlay, remember picture]
\node[anchor=center] at (current page.center) {
\begin{beamercolorbox}[center]{title}
     Phase III:\\\textbf{Solution Methods}
  \end{beamercolorbox}};
\end{tikzpicture}

\end{frame}
%%%%%%%%%%%%%%%%%%%%%%%%%%%%%%%%%
%%%%%%%%%%%%%%%%%%%%%%%%%%%%%%%%%%%
%%%%%%%%%%%%%%%%%%%%%%%%%%%%%%%%%%%%
%%%%%%%%%%%%%%%%%%%%%%%%%%%%%%%%%%%
%%%%%%%%%%%%%%%%%%%%%%%%%%%%%%%%%
\section{Solution Method}
\begin{frame}{Nonlinear Programming (NLP)}
\begin{block}{How do we solve Optimal Power Flow?}
\begin{enumerate}[I.]
\item<1-> Interior Point Method
\item<2-> Gradient Decent Method
\item<3-> Heuristic Methods
\end{enumerate}
\end{block}
\onslide<4>{\begin{alertblock}
{Interior Point Method}
The focus of this study is the interior point method solution
\end{alertblock}}
\end{frame}
%%%%%%%%%%%%%%%%%%%%%%%%%%%%%%%%%
%%%%%%%%%%%%%%%%%%%%%%%%%%%%%%%%%%%
%%%%%%%%%%%%%%%%%%%%%%%%%%%%%%%%%%%%
%%%%%%%%%%%%%%%%%%%%%%%%%%%%%%%%%%%
%%%%%%%%%%%%%%%%%%%%%%%%%%%%%%%%%

%%%%%%%%%%%%%%%%%%%%%%%%%%%%%%%%%
%%%%%%%%%%%%%%%%%%%%%%%%%%%%%%%%%%%
%%%%%%%%%%%%%%%%%%%%%%%%%%%%%%%%%%%%
%%%%%%%%%%%%%%%%%%%%%%%%%%%%%%%%%%%
%%%%%%%%%%%%%%%%%%%%%%%%%%%%%%%%%

\begin{frame}

\begin{center}
\begin{tabular}{|l|}
\hline
\rowcolor{yellow}
Step(1): Applying Slack variables and the barrier term: \\
\rowcolor{Gray}

$\min_{\mathbf{X}} \bigg[F(\mathbf{X})-\gamma\sum_{i=1}^{N_h}{ln(z_{i})}\bigg]$\\ \rowcolor{Gray}

 $\textrm{s.t. }  \mathbf{G}(\mathbf{X})=0$\\ \rowcolor{Gray}

$\mathbf{H}(\mathbf{X})+\mathbf{Z}=0$\\ \rowcolor{Gray}

 $\mathbf{Z}\geq 0$\\
\hline
slack variables "Z" convert inequality constraints\\
 to equality constraints.\\
\hline
\end{tabular}
\end{center}

\end{frame}

%%%%%%%%%%%%%%%%%%%%%%%%%%%%%%%%%
%%%%%%%%%%%%%%%%%%%%%%%%%%%%%%%%%%%
%%%%%%%%%%%%%%%%%%%%%%%%%%%%%%%%%%%%
%%%%%%%%%%%%%%%%%%%%%%%%%%%%%%%%%%%
\begin{frame}
\begin{center}
\begin{tabular}{|l l|}
\hline
\rowcolor{cyan} \multicolumn{2}{|l|}{Step (2): Calculate and form the Lagrangian of Barrier subproblem} \\
\rowcolor{Gray} \multicolumn{2}{|l|}{ $\boldsymbol{\mathcal{L}}^{\gamma}(\mathbf{X},\mathbf{Z},\boldsymbol{\lambda},\boldsymbol{\mu})=f(\mathbf{X})+\boldsymbol{\lambda}^\top \mathbf{G}(\mathbf{X})
+\boldsymbol{\mu}^\top(\mathbf{H}(\mathbf{X})+\mathbf{Z})-\gamma \sum_{i=1}^{N_g}{ln(z_{i})}$}\\
\hline
\multicolumn{2}{l}{}\\
\multicolumn{2}{l}{}\\
\hline
\rowcolor{magenta}
\multicolumn{2}{|l|}{Step (3): Calculate the KKT\footnote{Karush–Kuhn–Tucker conditions} of the Lagrangian}\\
\rowcolor{Gray}
\begin{tabular}[l]{@{}l@{}}Step (3)\\$KKT_X$  \end{tabular} &$\boldsymbol{\mathcal{L}}_{\mathbf{X}}^{\gamma}(\mathbf{X},\mathbf{Z},\boldsymbol{\lambda},\boldsymbol{\mu})=f_{\mathbf{X}}+\boldsymbol{\lambda}^\top \mathbf{G}_{\mathbf{X}}+\boldsymbol{\mu}^\top \mathbf{H}_{\mathbf{X}}=0$\\
\rowcolor{Gray}
\begin{tabular}[l]{@{}l@{}}Step (3)\\$KKT_Z$  \end{tabular}&$\boldsymbol{\mathcal{L}}_{\mathbf{Z}}^{\gamma}(\mathbf{X},\mathbf{Z},\boldsymbol{\lambda},\boldsymbol{\mu})=\boldsymbol{\mu}^\top - \gamma \mathbf{e}^\top\mathbf{diag}(\mathbf{Z})^{-1}=0$\\
\rowcolor{Gray}
\begin{tabular}[l]{@{}l@{}}Step (3)\\$KKT_\lambda$   \end{tabular}  &$\boldsymbol{\mathcal{L}}_{\boldsymbol{\lambda}}^{\gamma}(\mathbf{X},\mathbf{Z},\boldsymbol{\lambda},\boldsymbol{\mu})=\mathbf{G}^\top (\mathbf{X})=0$
\\
  \rowcolor{Gray}
\begin{tabular}[l]{@{}l@{}}Step (3)\\$KKT_\mu$   \end{tabular}    & $
\boldsymbol{\mathcal{L}}_{\boldsymbol{\mu}}^{\gamma}(\mathbf{X},\mathbf{Z},\boldsymbol{\lambda},\boldsymbol{\mu})=\mathbf{H}^\top(\mathbf{X})+\mathbf{Z}^\top=0$\\
\hline
\end{tabular}
\end{center}


\end{frame}
%%%%%%%%%%%%%%%%%%%%%%%%%%%%%%%%%
%%%%%%%%%%%%%%%%%%%%%%%%%%%%%%%%%%%
%%%%%%%%%%%%%%%%%%%%%%%%%%%%%%%%%%%
%%%%%%%%%%%%%%%%%%%%%%%%%%%%%%%%%


\begin{frame}

\begin{center}
\begin{tabular}{|l l|}
\hline
\begin{tabular}[l]{@{}l@{}}Nonlinear \\Algebric \\Equations  \end{tabular} &$\boldsymbol{\Omega}(\mathbf{X},\mathbf{Z},\boldsymbol{\lambda},\boldsymbol{\mu})
=\begin{bmatrix}
    f_{\mathbf{X}}+\boldsymbol{\lambda}^\top \mathbf{G}_{\mathbf{X}}+\boldsymbol{\mu}^\top \mathbf{H}_{\mathbf{X}} \\
   \mathbf{diag}(\mathbf{Z})\boldsymbol{\mu}^\top - \gamma \mathbf{e}^\top\\
    \mathbf{G}^\top (\mathbf{X})\\
    \mathbf{H}^\top(\mathbf{X})+\mathbf{Z}^\top
\end{bmatrix}=0$\\

$S.t.$&$\quad \mathbf{Z} > 0$\\

&$\quad \boldsymbol{\mu} > 0$\\
\hline
\multicolumn{2}{l}{}\\
\hline
\rowcolor{green}
\multicolumn{2}{|l|}{Step (4): Apply Newton Raphson Method}\\
\rowcolor{Gray}
\multicolumn{2}{|l|}{ \quad $[\boldsymbol{\Omega}_\mathbf{X} \ \boldsymbol{\Omega}_\mathbf{Z} \ \boldsymbol{\Omega}_{\boldsymbol{\lambda}} \ \boldsymbol{\Omega}_{\boldsymbol{\mu}}]^k{[\Delta \mathbf{X} \ \Delta \mathbf{Z} \ \Delta \boldsymbol{\lambda} \ \Delta \boldsymbol{\mu}]^\top}^k=-\boldsymbol{\Omega}(\mathbf{X},\mathbf{Z},\boldsymbol{\lambda},\boldsymbol{\mu})^k$}\\
\hline
\end{tabular}
\end{center}

\end{frame}

%%%%%%%%%%%%%%%%%%%%%%%%%%%%%%%%%
%%%%%%%%%%%%%%%%%%%%%%%%%%%%%%%%%%%
\begin{frame}
\begin{center}
\begin{tabular}{|l l|}
\hline
\begin{tabular}[l]{@{}l@{}}Step (5): \\{Inverse Jacobian} \\ of Newton Raphson  \end{tabular} \cellcolor{red} & \cellcolor{Gray}${\textcolor{red}{\begin{colorbmatrix}
    \mathbf{M}&  \mathbf{G}_{\mathbf{X}}^\top\\
    \mathbf{G}_{\mathbf{X}} & 0\\
\end{colorbmatrix}}}^k
{\begin{colorbmatrix}
    \Delta \mathbf{X}\\
   \Delta\boldsymbol{\lambda}
\end{colorbmatrix}}^k=
{\begin{colorbmatrix}
    -\mathbf{N}\\
   -\mathbf{G}(\mathbf{X})
\end{colorbmatrix}}^k $\\
\hline
\end{tabular}
\end{center}
$\mathbf{M} \in \mathbb{R}^{N_x \times N_x}$ and $\mathbf{N} \in \mathbb{R}^{N_x \times 1}$ are defined as:


\begin{center}
\begin{tabular}{|l l|}
\hline
&$ \mathbf{M}=\boldsymbol{\mathcal{L}}_{\mathbf{X}\mathbf{X}}^{\gamma}+\mathbf{H}_{\mathbf{X}}^\top\mathbf{diag}(\mathbf{Z})^{-1}\mathbf{diag}(\boldsymbol{\mu}) \mathbf{H}_{\mathbf{X}}$\\

&$\mathbf{N}=f_{\mathbf{X}}^\top+\mathbf{G}_{\mathbf{X}}^\top\boldsymbol{\lambda}+\mathbf{H}_{\mathbf{X}}^\top\boldsymbol{\mu} +\mathbf{H}_{\mathbf{X}}^\top\mathbf{diag}(\mathbf{Z})^{-1}(\gamma \mathbf{e} +\mathbf{diag}(\boldsymbol{\mu})\mathbf{H}(\mathbf{X}))$\\

&$\boldsymbol{\mathcal{L}}_{\mathbf{X}\mathbf{X}}^{\gamma} = f_{\mathbf{XX}}+\mathbf{G}_{\mathbf{XX}}(\boldsymbol{\lambda})+\mathbf{H}_{\mathbf{XX}}(\boldsymbol{\mu})$\\
\hline
\end{tabular}
\end{center}

\end{frame}
%%%%%%%%%%%%%%%%%%%%%%%%%%%%%%%%%
%%%%%%%%%%%%%%%%%%%%%%%%%%%%%%%%%%%
%%%%%%%%%%%%%%%%%%%%%%%%%%%%%%%%%%%%
\begin{frame}{Iterations}
\vskip -1.5cm
\begin{block}{Successive iterations in Interior Point Method}
\begin{enumerate}[i.]
\item<1-> \textcolor{gray}{\textbf{Function Evaluations}} \only<1> {calculation of $\mathbf{G}_{\mathbf{X}}=\frac{\partial \mathbf{G}}{\partial \mathbf{X}}$, $\mathbf{H}_{\mathbf{X}}=\frac{\partial \mathbf{H}}{\partial \mathbf{X}}$, $F_{\mathbf{X}}=\frac{\partial F}{\partial \mathbf{X}}$, $\mathbf{G}_{\mathbf{X}\mathbf{X}}=\frac{\partial}{\partial \mathbf{X}}(\mathbf{G}_\mathbf{X}^\top \boldsymbol{\lambda})$, $\mathbf{H}_{\mathbf{X}\mathbf{X}}=\frac{\partial}{\partial \mathbf{X}}(\mathbf{H}_\mathbf{X}^\top \boldsymbol{\lambda})$, $F_{\mathbf{X}\mathbf{X}}=\frac{\partial}{\partial \mathbf{X}}({F}_\mathbf{X}^\top)$ in order to form coefficient matrix and right hand side of ${\begin{colorbmatrix}
    \mathbf{M}&  \mathbf{G}_{\mathbf{X}}^\top\\
    \mathbf{G}_{\mathbf{X}} & 0\\
\end{colorbmatrix}}^k
{\begin{colorbmatrix}
    \Delta \mathbf{X}\\
   \Delta\boldsymbol{\lambda}
\end{colorbmatrix}}^k=
{\begin{colorbmatrix}
    -\mathbf{N}\\
   -\mathbf{G}(\mathbf{X})
\end{colorbmatrix}}^k $}
\item<2-> \textcolor{mine1}{\textbf{Linear Algebraic Solver}} \only<2>{Calculate the Inverse ${\begin{colorbmatrix}
    \mathbf{M}&  \mathbf{G}_{\mathbf{X}}^\top\\
    \mathbf{G}_{\mathbf{X}} & 0\\
\end{colorbmatrix}}^k$ in ${\begin{colorbmatrix}
    \mathbf{M}&  \mathbf{G}_{\mathbf{X}}^\top\\
    \mathbf{G}_{\mathbf{X}} & 0\\
\end{colorbmatrix}}^k
{\begin{colorbmatrix}
    \Delta \mathbf{X}\\
   \Delta\boldsymbol{\lambda}
\end{colorbmatrix}}^k=
{\begin{colorbmatrix}
    -\mathbf{N}\\
   -\mathbf{G}(\mathbf{X})
\end{colorbmatrix}}^k $}
\item<3-> \textcolor{mine2}{\textbf{Miscellaneous}} \only<3>{ Computational time for other components of IP such as step control and step update: $X^{k+1}=X^k+\Delta X $}
\item<4-> Bottleneck of IP:
\only<4>{
\begin{backgroundblock}{20mm}{50mm}
        \includegraphics[width=3.5 in , height=1.5 in]{Figures/IPbottleneck.png}
        \end{backgroundblock}}
\end{enumerate}
\end{block}
\end{frame}
%%%%%%%%%%%%%%%%%%%%
%%%%%%%%%%%%%%%5 
%%%%%%%%%%%%%%%%%%%%
%%%%%%%%%%%%%%%5
\begin{frame}[plain]
\begin{tikzpicture}[overlay, remember picture]
\node[anchor=center] at (current page.center) {
\begin{beamercolorbox}[center]{title}
     Phase IV:\\\textbf{Speed-up}
  \end{beamercolorbox}};
\end{tikzpicture}

\end{frame}
%%%%%%%%%%%%%%%%%%%%%%%%%%%%%%%%%%%
%%%%%%%%%%%%%%%%%%%%%%%%%%%%%%%%%
%%%%%%%%%%%%%%%%%%%%%%%%%%%%%%%%%
%%%%%%%%%%%%%%%%%%%%%%%%%%%%%%%%%%%
%%%%%%%%%%%%%%%%%%%%%%%%%%%%%%%%%%%%
%%%%%%%%%%%%%%%%%%%%%%%%%%%%%%%%%%%
%%%%%%%%%%%%%%%%%%%%%%%%%%%%%%%%%
\section{Speedup}
\subsection{Sparsity}
\begin{frame}
\begin{figure}[!htbp]
\centering
\includegraphics[width=4.1 in , height=2.8 in]{Figures/Sparsity.png}
\caption{Sparse stracture of the Newton-Raphson Jacobian}
\label{sparsity}
\end{figure}
\end{frame}

%%%%%%%%%%%%%%%%%%%%%%%%%%%%%%%%%
%%%%%%%%%%%%%%%%%%%%%%%%%%%%%%%%%%%
%%%%%%%%%%%%%%%%%%%%%%%%%%%%%%%%%%%%
%%%%%%%%%%%%%%%%%%%%%%%%%%%%%%%%%%%
\begin{frame}{Connectivity Matrices}
\begin{center}
\begin{tabular}{|l l|}
\hline
{\tiny a term in pf equation:} & ${\begin{bmatrix} \mathbf{C}_g \end{bmatrix} \ \mkern-10mu}_{n_b \times n_g} {\begin{bmatrix}\mathbf{P}_g \end{bmatrix} \ \mkern-10mu}_{n_g \times 1} $\\
\hline
 \cellcolor{Gray}example: &  \cellcolor{Gray}$ {\textcolor{red}{{\begin{colorbmatrix}
    0&  0 &0 & 0 & 0 \\
    0&1 & 0 &0 &0\\
0 &0 &0 &0&0\\
    1&0 & 0 &0 &0\\
0 &0 &0 &0&0\\
0 &0 &1 &0&0\\
0 &0 &0 &0&0\\
0 &0 &0 &0&0\\
0 &0 &0 &1&0\\
0 &0 &0 &0&1
\end{colorbmatrix} \ \mkern-10mu}_{n_b \times n_g}}}
{\begin{colorbmatrix}
    \mathbf{P}_{g_1} \\
   \mathbf{P}_{g_2}\\ 
\mathbf{P}_{g_3}\\
 \mathbf{P}_{g_4}\\ 
\mathbf{P}_{g_5}\\
\end{colorbmatrix} \ \mkern-10mu}_{n_g \times 1}=
{\begin{colorbmatrix}
0\\
     \mathbf{P}_{g_2} \\
0\\
   \mathbf{P}_{g_1}\\ 
0\\
\mathbf{P}_{g_3}\\
0\\
0\\
 \mathbf{P}_{g_4}\\ 
\mathbf{P}_{g_5}\\
\end{colorbmatrix} \ \mkern-10mu}_{n_b \times 1}$ \\
\hline
\end{tabular}
\end{center}
\end{frame}
%
%%%%%%%%%%%%%%%%%%%%
%%%%%%%%%%%%%%%5
%%%%%%%%%%%%%%%%%%%%
%%%%%%%%%%%%%%%5

\section{Speed-up the solution proposal}
\subsection{Reordering}
\begin{frame}
\begin{figure}[!htbp]
\centering
\includegraphics[width=3.4 in , height=1.7 in]{Figures/reorder.png}
% where an .eps filename suffix will be assumed under latex,
% and a .pdf suffix will be assumed for pdflatex; or what has been declared
% via \DeclareGraphicsExtensions.
\caption{Structure of Jacobian of the Newton-Raphson's algorithm before and after reordering.}
\label{fig:1}\vspace*{-0.4cm}
\end{figure}
\begin{alertblock}{Jacobian of Newton-Raphson}
\centering
{\tiny
${\textcolor{red}{\begin{colorbmatrix}
    \mathbf{M}&  \mathbf{G}_{\mathbf{X}}^\top\\
    \mathbf{G}_{\mathbf{X}} & 0\\
\end{colorbmatrix}}}
{\begin{colorbmatrix}
    \Delta \mathbf{X}\\
   \Delta\boldsymbol{\lambda}
\end{colorbmatrix}}=
{\begin{colorbmatrix}
    -\mathbf{N}\\
   -\mathbf{G}(\mathbf{X})
\end{colorbmatrix}} $\\
The solution is published in Papers I and II of this thesis.}
\end{alertblock}
 \end{frame}
%%%%%%%%%%%%%%%%%%%%%%%%%%%%%%%%%
%%%%%%%%%%%%%%%%%%%%%%%%%%%%%%%%%%%
%%%%%%%%%%%%%%%%%%%%%%%%%%%%%%%%%%%%
%%%%%%%%%%%%%%%%%%%%%%%%%%%%%%%%%%%
%%%%%%%%%%%%%%%%%%%%%%%%%%%%%%%%%
%%%%%%%%%%%%%%%%%%%%%%%%%%%%%%%%%
%%%%%%%%%%%%%%%%%%%%%%%%%%%%%%%%%%%
%%%%%%%%%%%%%%%%%%%%%%%%%%%%%%%%%%%%
%%%%%%%%%%%%%%%%%%%%%%%%%%%%%%%%%%%
%%%%%%%%%%%%%%%%%%%%%%%%%%%%%%%%%
\section{Results}
\subsection{Schur-Complement}
\begin{frame}
\begin{figure}[!htbp]
\centering
\includegraphics[width=3.0 in , height=2.5 in]{Figures/SCLU118Bus.png}
\caption{Total time ($\mathrm{TotalTime= No_{\cdot} of \ Iter_{\cdot} \times TimePerIter}$) for solution of the linear KKT systems, Case: IEEE 118.}
\label{118bus}
\end{figure}
\end{frame}

%%%%%%%%%%%%%%%%%%%%%%%%%%%%%%%%%
%%%%%%%%%%%%%%%%%%%%%%%%%%%%%%%%%%%
%%%%%%%%%%%%%%%%%%%%%%%%%%%%%%%%%%%%
%%%%%%%%%%%%%%%%%%%%%%%%%%%%%%%%%%%
%%%%%%%%%%%%%%%%%%%%%%%%%%%%%%%%%

\begin{frame}{Analytical Derivatives}
\vskip -0.6cm
\begin{table}
\tiny
\begin{center}
 \caption{Total time ($\mathrm{TotalTime= No_{\cdot} of \ Iter_{\cdot} \times TimePerIter}$) elapsed to calculate: 1) Analytical (hand-coded) derivatives, and 2) Numerical derivatives }
\label{tab:numericVSanalytic}
\begin{threeparttable}
\begin{tabularx}{\textwidth}{m s s s s c s s s c m }
\toprule
&&&&   \multicolumn{3}{c}{Analytical}&&\multicolumn{3}{c}{Numerical} \\
\cmidrule{5-7}  \cmidrule{9-11}
Case     & $T$&$n_y$ &iter & ${F}_\mathbf{X}$(s)&$\mathbf{G}_\mathbf{X}$+ $\mathbf{H}_\mathbf{X}$(s)&$\boldsymbol{\mathcal{L}}_{\mathbf{X}\mathbf{X}}^{\gamma}$(s)& &${F}_\mathbf{X}$(s)&$\mathbf{G}_\mathbf{X}$+ $\mathbf{H}_\mathbf{X}$(s)&$\boldsymbol{\mathcal{L}}_{\mathbf{X}\mathbf{X}}^{\gamma}$(s) \\
\midrule
Case9     &2 &5&13& 0.03 & 0.13 & 0.14 & &0.43 & 0.98 &140.07\\
Case9     &10&5&23& 0.08  & 0.36 & 0.37 & &11.32 & 30.62  & 22815.29\\
IEEE30    &2&5&12 & 0.04 & 0.25 & 0.18     &   & 1.01  & 2.16  & 682.70\\
IEEE30    &10&5&16& 0.05 & 0.24 & 0.25 && 16.73 & 49.12  & 79712.78\\
IEEE118   &2&5&22 & 0.04 & 0.19 & 0.20   & &7.41 & 18.07 & 24557.09\\
IEEE118   &10&5&37&  0.09 & 0.62 & 0.82   & & 158.21\tnote{1}  & 572.09\tnote{1} & 4599735\tnote{1} \\
\begin{tabular}[l]{@{}l@{}}Pegase\\1354  \end{tabular} &2 &5&23& 0.05 & 0.61 & 0.78    & &85.54\tnote{1}  & 496.18\tnote{1} & 7185888\tnote{1}  \\
\begin{tabular}[l]{@{}l@{}}Pegase \\1354  \end{tabular}&10&5&33& 0.10 & 3.77 & 5.15  & & 588.06\tnote{1}  & 3530\tnote{1} & 51550941\tnote{1}  \\
\bottomrule
\end{tabularx}
\begin{tablenotes}
\item[1] {Estimated total time: The time elapsed for one iteration multiplied to the iteration that would take to converge}
\end{tablenotes}
\end{threeparttable}
\end{center}
\end{table}
\begin{alertblock}{Test Cases}
Test cases of Case9, IEEE30, IEEE118, and PEGASE1354 are part of open source MATPOWER library.
\end{alertblock}
\end{frame}

%%%%%%%%%%%%%%%%%%%%%%%%%%%%%%%%%
%%%%%%%%%%%%%%%%%%%%%%%%%%%%%%%%%%%
%%%%%%%%%%%%%%%%%%%%%%%%%%%%%%%%%%%%
%%%%%%%%%%%%%%%%%%%%%%%%%%%%%%%%%%%
%%%%%%%%%%%%%%%%%%%%%%%%%%%%%%%%%
%%%%%%%%%%%%%%%%%%%%%%%%%%%%%%%%%%%%%%%%%%
%%%%%%%%%%%%%%%%%%%%%%%%%%%%%%%%%%%%%%%%%%
%%%%%%%%%%%%%%%%%%%%%%%%%%%%%%%%%%%%%%%%%%
%%%%%%%%%%%%%%%%%%%%%%%%%%%%%%%%%%%%%%%%%%


\begin{frame}[plain]
\begin{tikzpicture}[overlay, remember picture]
\node[anchor=center] at (current page.center) {
\begin{beamercolorbox}[center]{title}
     Phase V:\\\textbf{Future Works}
  \end{beamercolorbox}};
\end{tikzpicture}

\end{frame}

%%%%%%%%%%%%%%%%%%%%%%%%%%%%%%%%% 
%%%%%%%%%%%%%%%%%%%%%%%%%%%%%%%%%%% 
%%%%%%%%%%%%%%%%%%%%%%%%%%%%%%%%%%%% 
%%%%%%%%%%%%%%%%%%%%%%%%%%%%%%%%%%% 
%


\begin {frame}{Future work}
\begin{block}{{Goals}}
\begin{enumerate}[i.]
\item <1->{Translate my entire setup to C and C++.
\begin{itemize}
\item > Input reading
\item > Sparsity libraries made particularly for ACOPF and MPOPF mathematical calculations.
\item > Functions to calculate the first and second order gradients.
\item > Rebuilt a linear customized solver for this purpose.
\end{itemize}
}
\item <2-> Benchmark the entire setup with state of the art Nonlinear Solvers.
\end{enumerate}
\end{block}
\end{frame}
%%%%%%%%%%%%%%%%%%%%%%%%%%%%%%%%% 
%%%%%%%%%%%%%%%%%%%%%%%%%%%%%%%%%%% 
%%%%%%%%%%%%%%%%%%%%%%%%%%%%%%%%%%%% 
%%%%%%%%%%%%%%%%%%%%%%%%%%%%%%%%%%% 
%
\begin{frame}
\centering
Thank you for your attention!\\
		\vskip 0.8cm

\centering
\includegraphics[scale=0.2]{ntnulogo_eng.png}
\end{frame} 

\end{document}



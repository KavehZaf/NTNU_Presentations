%%%%% 26-02-2022 package multicol multirow
\newcommand{\mc}[2]{\multicolumn{#1}{c}{#2}}
\definecolor{Gray}{gray}{0.85}
\definecolor{LightCyan}{rgb}{0.88,1,1}
\usepackage{hhline,longtable}
\usepackage{tabularx,booktabs}

\usepackage{multicol}
\usepackage{multirow}
%% 26-02-2022
\usepackage{hhline,longtable}
\usepackage{threeparttable}


\usepackage{enumerate}
\usepackage[shortlabels]{enumitem}

\setlist[enumerate, 1]{label =\textbf{\arabic*.}}
\setlist[enumerate, 2]{label =\textbf{\theenumi \alph*}}
\usepackage{array,multirow}

\usepackage{amsmath,mathtools}

\usepackage{amssymb}
\usepackage{amsthm}
\usepackage{mathtools}



\DeclarePairedDelimiter\abs{\lvert}{\rvert}%
\DeclarePairedDelimiter\norm{\lVert}{\rVert}%

% Swap the definition of \abs* and \norm*, so that \abs
% and \norm resizes the size of the brackets, and the 
% starred version does not.
\makeatletter
\let\oldabs\abs
\def\abs{\@ifstar{\oldabs}{\oldabs*}}


\newsavebox{\bmatrixbox}
\newenvironment{colorbmatrix}
  {\begin{lrbox}{\bmatrixbox}
   \mathsurround=0pt
   $\displaystyle
   \begin{bmatrix}}
  {\end{bmatrix}$%
   \end{lrbox}%
   \usebox{\bmatrixbox}%
   \kern-\wd\bmatrixbox
   \makebox[0pt][l]{$\left[\vphantom{\usebox{\bmatrixbox}}\right.$}%
   \kern\wd\bmatrixbox
}
\usepackage[linesnumbered,ruled,vlined]{algorithm2e}

\usepackage[utf8]{inputenc}
\newcommand{\tabitem}{~~\llap{\textbullet}~~}


%27-02-2022%%%%%%%%%%%%%%%%%%%%%%%%%%%%%%%%%%%
%%%%%%%%%%%%%%%%%%%%%%%%%%%%%%%%%
%%%%my colors%%%%%%%%%%%%%%%%%%%%%%%%

\definecolor{mine1}{RGB}{255, 128, 0}
\definecolor{mine2}{RGB}{255, 202, 23}





\usepackage{filecontents}% to embed the file `myreferences.bib` in your `.tex` file

\makeatletter
\newcommand\footnoteref[1]{\protected@xdef\@thefnmark{\ref{#1}}\@footnotemark}
\makeatother


\newcolumntype{C}{>{\centering\arraybackslash}X} % centered version of "X" type
\setlength{\extrarowheight}{1pt}

 \newcolumntype{b}{>{\centering\arraybackslash\hsize=2.3\hsize}X}
\newcolumntype{s}{>{\centering\arraybackslash\hsize=.45\hsize}X}
\newcolumntype{m}{>{\centering\arraybackslash\hsize=.9\hsize}X}


%03-03-2022
\usepackage{emoji}



%%%% animation package 21-09-2021
\usepackage{animate}


%%%%%%%%%%%%%%%%%09-09-2021\\\ copied from Webinarinnovation page
% Here I would like to make a new command to change the transparency of a photo and put it as a background photo
\usepackage{tikz}



%%%%%%%%%%%% 22-10-2020 Background block package
% beamer: How to place images behind text (z-order)
% (http://tex.stackexchange.com/a/134311)
\makeatletter
\newbox\@backgroundblock
\newenvironment{backgroundblock}[2]{%
  \global\setbox\@backgroundblock=\vbox\bgroup%
    \unvbox\@backgroundblock%
    \vbox to0pt\bgroup\vskip#2\hbox to0pt\bgroup\hskip#1\relax%
}{\egroup\egroup\egroup}
\addtobeamertemplate{background}{\box\@backgroundblock}{}
\makeatother

%%%%%%%%%%%%%%%%%%%%%%%%%%%%%%%%%%%%%%%%%%%%
%%%%%%%%%%%26-10-2020% set figure number
\setbeamertemplate{caption}[numbered]


%%%%%%%%%%%%%%%%26-10-2020
\usepackage{amssymb,amsmath}


%%%%%%%%%%%%%%%%26-10-2020
\newenvironment{variableblock}[3]{%
  \setbeamercolor{block body}{#2}
  \setbeamercolor{block title}{#3}
  \begin{block}{#1}}{\end{block}}
  
  \setbeamercolor{block body alerted}{bg=alerted text.fg!10}
\setbeamercolor{block title alerted}{bg=alerted text.fg!20}
\setbeamercolor{block body}{bg=structure!10}
\setbeamercolor{block title}{bg=structure!20}
\setbeamercolor{block body example}{bg=green!10}
\setbeamercolor{block title example}{bg=green!20}

\setbeamertemplate{blocks}[rounded][shadow=true]


%%%%%%%%%%%%%%%%%%%%%26-10-2020
\setbeamertemplate{footline}[frame number]

%%%%%%%%%%%% 26-10-2020
\usepackage{color, colortbl}
\definecolor{Gray}{gray}{0.85}

%%%%%%%%%%%%%%%%%%27-10-2020
\usepackage[T1]{fontenc}


